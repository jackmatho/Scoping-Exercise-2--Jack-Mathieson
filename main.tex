\documentclass{article}
\usepackage[utf8]{inputenc}

\title{Scoping Exercise 2- Jack Mathieson}
\author{jack.mathieson }
\date{August 2019}

\begin{document}

\maketitle
There are two processes in ethnographic research that I have identified as taking up time. Doing ethnographic research involves a lot of research-by-participation and then later reflection and compiling notes, where the bulk of novel research is done. Two necessary components of this process are prior reading and transcribing interviews. The unfortunate side to these components is that they take up just as much, if not more time than the actual research-by-participation. This scoping exercise hopes to break down these two components into manageable tasks that  utilize technology to get results.
\section*{Decomposition}
The following steps attempt to break down the pains in this format way: This is what I'm doing, what that means is that to solve it I have to do/I need... x, y, z.
\subsection*{\textit{Pain 1: Wasting time on finding relevant material}}
-Searching for sources (articles, books, journals, websites etc.) on a particular topic/s

-Analysing source to see if it matches the search at a deeper level beyond title, abstract or key words.

-Identifying the field area that the source has been allocated to in order to narrow the search for further sources on the same topic.

\subsection*{\textit{Pain 2: Sifting through recorded dialogue to find relevant information or a pattern of discussion}}
-Quality recording of the dialogue that captures at least the audible aspect of the discussion.

-A tool for recording the metadata of each recording, providing information on time, place, location and participant information.

-A method of isolating specific words from a flow of sound.

-A tool to break a continuous flow of sound into discernible parts of a discussion.

-A tool that can isolate multiple spoken words from surrounding sounds and then display those words AND a way of communicating null data when the word is not discerned.

\section*{Pattern Recognition (responding to pains)}
This section will focus in on what it is that is being accomplished if I actually complete the many steps in the Decomposition of my pains. Basically, if I successfully completed steps x, y and z then I have essentially been doing the following.
\subsection*{\textit{Patterns of Pain 1:}}
-Building a collection of search terms that correlate to a given topic

-Discerning the content of a source by way of its word use

-Finding connections between the search terms and the words contained in a source

\subsection*{\textit{Patterns of Pain 2:}}
-Capturing a flow of sound via recording

-Breaking up a flow of recorded sound into discernible sections based on various criteria (such as time or theme)

-Treating words as a pattern in a flow of sound to be isolated and identified.
\section*{Algorithm Design}
Finally, the processes that have been outlined in Pattern Recognition will be written here as instructions that (given the correct coding and soft wear) a computer could conceivably follow.

\subsection*{\textit{Instructions for Pattern 1:}}
-Search database for sources who's titles, names and words within abstracts  correlate to a list of search terms provided

-Once a source is found read through the text and find all the most common words used, excluding the following list of grammatical and syntax terms (e.g I, and, but, however)(any further words that are deemed unnecessary can be added to this list of excluded words)

-Once each source is examined, add the most common word from the source to a list of words that are connected with the given search term (each search term will now have a list of words that correspond to the term).

Essentially what this accomplishes is that whenever a search term is used a list of related terms will accompany it based on how often they are used within the articles, journals or books that the search term brought up. E.g: if I want to know about fruitcakes, the related words will likely be the ingredients. If I want to know about Nazi Germany, the word Hitler will come up next to it. This narrows down the search process by providing further, more specific and related search terms.
\subsection*{\textit{Instructions for Pattern 2:}}
-Record all audible sound chronologically in one stream. Provide a title section that records the name, place, date and other metadata of the audio file.

-Separate the stream into segments of time as small as possible (e.g 0.1 second segments) that can be cut, copied and shuffled. ALWAYS retain a copy of the original audio stream regardless of how many editions are made.

-Scan the stream for audio patterns that correspond to a list of audio patterns that signify a list of words. Transcribe the words and show where along the stream these words occur.

The purpose of this is to break down a long flow of words and sounds into a manageable dialogue that can be segmented for closer analysis of each part. Whichever parts are then relevant or irrelevant can then be identified at the discretion of the researcher.
\end{document}
